\documentclass[12pt,reqno,oneside]{pucthesis}

%\draft

\usepackage[left=4cm, top=3cm, bottom=4cm, right=2.5cm]{geometry}
\usepackage[english]{babel}
\usepackage{subfig}
\usepackage{fancybox}
\usepackage{float}
\usepackage[T1]{fontenc}
\usepackage{numprint}
\npthousandsep{,}

% Uncomment the following to use Times fonts, as required by the OGRS
%\usepackage{mathptmx}
\usepackage[usenames]{color}
\usepackage[colorinlistoftodos]{todonotes}
\usepackage[utf8]{inputenc}
\usepackage{pdfpages}
% \usepackage{hyperref}
% \usepackage{apacite}
%
\hypersetup{
	colorlinks=true,
	linkcolor=black,
	citecolor=black,
	filecolor=black,
	urlcolor=black
}

\usepackage{booktabs}
\usepackage{times}
\usepackage{graphicx}
\usepackage{epic}
\usepackage{eepic}
\usepackage{url}
\usepackage{algorithm2e}
\usepackage{rotfloat}
\usepackage{xspace}
\usepackage{amsmath}
\usepackage{amssymb}
\usepackage{stmaryrd}
\usepackage{cases}
\usepackage{tikz}
 \usetikzlibrary{trees}
 \usetikzlibrary{shapes}
 \usetikzlibrary{shapes.geometric,shapes}
 \usetikzlibrary{positioning}
 \usetikzlibrary{calc}
 \usetikzlibrary{arrows}
 \usetikzlibrary{shadows}
\usepackage{tikz-qtree,tikz-qtree-compat}

           %%%%%%%%%%%%%%%%%%%%%%%%%%%%%%%%%%%%%%%%%%%%%%%%%%%%
           %   Preamble                                       %
           %------------------------------------------------- %
           %        \newcommand\...{...}                      %
           %        \newtheorem{}{}[]                         %
           %        \numberwithin{}{}                         %
           %%%%%%%%%%%%%%%%%%%%%%%%%%%%%%%%%%%%%%%%%%%%%%%%%%%%


%--------- NEW ENVIRONMENTS ---------
\newtheorem{definition}{\bf Definition}[chapter]
\newtheorem{property}{Property}[chapter]
\newtheorem{claim}{Claim}[chapter]
\newtheorem{lemma}{\bf Lemma}[chapter]
\newtheorem{proposition}{Proposition}[chapter]
\newtheorem{theorem}{\noindent \bf Theorem}[chapter]
\newtheorem{corollary}{\bf Corollary}[chapter]
\newtheorem{pf}{Proof}[chapter]
\newtheorem{example}{\bf Example}[chapter]
\newtheorem{remark}{Remark}[chapter]

%--------- PLACE ADDITIONAL ENVIRONMENTS/DEFINITIONS HERE ---------


%----------------------------------------------------------------------%
\begin{document}

           %%%%%%%%%%%%%%%%%%%%%%%%%%%%%%%%%%%%%%%%%%%%%%%%%%%%
           %                                                  %
           %  INITIALISATIONS : Top Matter                    %
           %                                                  %
           %%%%%%%%%%%%%%%%%%%%%%%%%%%%%%%%%%%%%%%%%%%%%%%%%%%%

\title[Optimization of semantic web queries]{Optimization of semantic
  web queries using SPARQL pattern trees}

\author{Andr\'es Ignacio Letelier Nagel}
%
\address{Escuela de Ingenier\'ia\\
         Pontificia Universidad Cat\'olica de Chile\\
         Vicu\~na Mackenna 4860\\
         Santiago, Chile\\
         {\it Tel.\/} : 56 (2) 354-2000}
\email{letelier@gmail.com}
%
\facultyto    {the School of Engineering}
\department   {}
\faculty      {Faculty of Engineering}
\degree       {Master of Science in Engineering}
              % or {Doctor in Engineering Sciences}
\advisor      {Marcelo Arenas S.}
\committeememberA {Juan L. Reutter D.}
%\committeememberB {Committee Member B (Optional)}
\guestmemberA {Jorge P\'erez R.}
%\guestmemberB {Guest Committee Member B (Optional)}
\ogrsmember   {Andrés Guesalaga M.}
\subject      {Engineering}
\date         {September 2013}
\copyrightname{Andr\'es Letelier Nagel}
\copyrightyear{MMXIII}

\dedication   {
To Andrés Letelier, for putting this template on github
}

           %%%%%%%%%%%%%%%%%%%%%%%%%%%%%%%%%%%%%%%%%%%%%%%%%%%%
           %   PRELIMINARIES                                  %
           %--------------------------------------------------%
           %      page i & ii: cover page                     %
           %      page iii: dedication                        %
           %%%%%%%%%%%%%%%%%%%%%%%%%%%%%%%%%%%%%%%%%%%%%%%%%%%%

\NoChapterPageNumber           % no header - footer on first page of chapter
\pagenumbering{roman}
\maketitle


           %%%%%%%%%%%%%%%%%%%%%%%%%%%%%%%%%%%%%%%%%%%%%%%%%%%%
           %   EXTRA PAGES                                    %
           %--------------------------------------------------%
           %      page --: not used                           %
           %%%%%%%%%%%%%%%%%%%%%%%%%%%%%%%%%%%%%%%%%%%%%%%%%%%%

%\newpage
%\thispagestyle{empty}

%----------------------------------------------------------------------%

           %%%%%%%%%%%%%%%%%%%%%%%%%%%%%%%%%%%%%%%%%%%%%%%%%%%%
           %      page iv: ACKNOWLEDGEMENTS                   %
           %%%%%%%%%%%%%%%%%%%%%%%%%%%%%%%%%%%%%%%%%%%%%%%%%%%%

\chapter*{Acknowledgements}
Acknowledgements


\cleardoublepage % In double-sided printing style makes the next page
                 % a right-hand page, (i.e. a truly odd-numbered page
                 % with respect to absolut counting), producing a blank
                 % page if necessary. Added by MTT 20.AUG.2002

%----------------------------------------------------------------------%

           %%%%%%%%%%%%%%%%%%%%%%%%%%%%%%%%%%%%%%%%%%%%%%%%%%%%
           %      page v & up ---                             %
           %            Table of Contents                     %
           %            List of Figures                       %
           %            List of Tables                        %
           %%%%%%%%%%%%%%%%%%%%%%%%%%%%%%%%%%%%%%%%%%%%%%%%%%%%

\tableofcontents
\listoftables
\listoffigures

\cleardoublepage % In double-sided printing style makes the next page
                 % a right-hand page, (i.e. a truly odd-numbered page
                 % with respect to absolut counting), producing a blank
                 % page if necessary. Added by MTT 20.AUG.2002


%----------------------------------------------------------------------%

           %%%%%%%%%%%%%%%%%%%%%%%%%%%%%%%%%%%%%%%%%%%%%%%%%%%%
           %      page x & xi: ABSTRACT - RESUMEN
           %%%%%%%%%%%%%%%%%%%%%%%%%%%%%%%%%%%%%%%%%%%%%%%%%%%%

\chapter*{Abstract}
\label{ch:abstract}
Abstract


           %%%%%%%%%%%%%%%%%%%%%%%%%%%%%%%%%%%%%%%%%%%%%%%%%%%%
           %   KEYWORDS                                       %
           %--------------------------------------------------%
           %      at the end of the abstract page             %
           %%%%%%%%%%%%%%%%%%%%%%%%%%%%%%%%%%%%%%%%%%%%%%%%%%%%


~\vfill
{\bf Keywords:} \parbox[t]{.75\textwidth}{
  SPARQL, RDF, Semantic Web, optimization, rewriting, database models
}


\chapter*{Resumen}
\label{ch:resumen}
Resumen


           %%%%%%%%%%%%%%%%%%%%%%%%%%%%%%%%%%%%%%%%%%%%%%%%%%%%
           %   PALABRAS CLAVES                                %
           %--------------------------------------------------%
           %      al final de la pagina de resumen            %
           %%%%%%%%%%%%%%%%%%%%%%%%%%%%%%%%%%%%%%%%%%%%%%%%%%%%

~\vfill
{\bf Palabras Claves:} \parbox[t]{.75\textwidth}{
  SPARQL, RDF, Web sem\'antica, optimizaci\'on, reescritura, modelos de bases de datos
}


\cleardoublepage % In double-sided printing style makes the next page
                 % a right-hand page, (i.e. a truly odd-numbered page
                 % with respect to absolut counting), producing a blank
                 % page if necessary. Added by MTT 20.AUG.2002

%======================================================================%

           %%%%%%%%%%%%%%%%%%%%%%%%%%%%%%%%%%%%%%%%%%%%%%%%%%%%
           %   TEXT  OF THESIS
           %%%%%%%%%%%%%%%%%%%%%%%%%%%%%%%%%%%%%%%%%%%%%%%%%%%%
\pagenumbering{arabic}


\chapter{Introduction}
\label{chapter:introduction}
Introduction



\chapter{Preliminaries}
\label{chapter:preliminaries}
Preliminaries



\chapter{Blabla}
\label{chapter:blabla}
Blablabla



\chapter{Conclusions and future research}
\label{chapter:conclusions}
This changes everything!




           %%%%%%%%%%%%%%%%%%%%%%%%%%%%%%%%%%%%%%%%%%%%%%%%%%%%
           %   REFERENCES
           %%%%%%%%%%%%%%%%%%%%%%%%%%%%%%%%%%%%%%%%%%%%%%%%%%%%

%\nocite{*} % To make all the uncited references to appear in the bibliography.
\bibliographystyle{apacite}


{\setlength{\baselineskip}{0.8\baselineskip}
\bibliography{biblio}
\par}

           %%%%%%%%%%%%%%%%%%%%%%%%%%%%%%%%%%%%%%%%%%%%%%%%%%%%
           %   APPENDICES
           %%%%%%%%%%%%%%%%%%%%%%%%%%%%%%%%%%%%%%%%%%%%%%%%%%%%

\newif\ifappendix
\appendixtrue
\ifappendix
\appendix
\clearpage
\begingroup%
\makeatletter%
\let\clearpage\relax% Stop LaTeX from going to a new page; and
\vspace*{\fill}%
\vspace*{\dimexpr-50\p@-\baselineskip}% Remove the initial (default) 50pt gap (plus 1 line)
\chapter*{APPENDIX}
\vspace*{\fill}%
\endgroup

\chapter{ADDITIONAL PROOFS}


\fi

           %%%%%%%%%%%%%%%%%%%%%%%%%%%%%%%%%%%%%%%%%%%%%%%%%%%%
           %   INDEX
           %%%%%%%%%%%%%%%%%%%%%%%%%%%%%%%%%%%%%%%%%%%%%%%%%%%%

%% Uncomment the following lines to include an index.

%% INSERT INDEX PAGE # IN TOC
%%%\addtocounter{chapter}{1}
%%%\addcontentsline{toc}{chapter}{\protect\numberline{\thechapter}{Index}}
%%\addcontentsline{toc}{chapter}{\protect\numberline{}{Index}}
%% NOTE: Insert "\label{IDX}" in '.ind' file after compiling the index
%% with makeindex.
%%\index{ @\label{IDX}}
%% The above NOTE is not really needed as can be achieved by
%% the trick below.
%\addtocounter{page}{1}
%\label{IDX}
%\addtocounter{page}{-1}
%\printindex

%----------------------------------------------------------------------%

\end{document}
